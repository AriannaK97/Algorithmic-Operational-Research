\documentclass[10pt]{article}
\usepackage{amsmath}
\usepackage{amssymb}
\usepackage[utf8]{inputenc}
\usepackage{enumitem}
\usepackage[english]{babel}
\newcommand{\R}{\mathbb{R}}

\title{Algorithmic Operation Research}
\date{18-10-2019}
\author{Theodora Panagea - 1115201400135 \\ Anna-Aikaterini Kavvada - 1115201500050}

\begin{document}
	\maketitle{}
  	\pagenumbering{arabic}
  	
%----------------------------Exercise 1 ------------------------
  	\begin{center}
  	\section{}
  	\end{center}\par
  	Let  $C \subseteq \R^n$ be a convex set with $x_1 , . . . x_k \in C$ and let $			\theta_1$ , . . . ,$\theta_k \in \R $
  	satisfy $\theta_i \geq 0$ and $\theta_1 + . . . + \theta_k = 1 $. Show that $\theta_1 x_1 + . . . + \theta_k x_k \in C $.\\
  	\textbf{Solution:}\par
  	We will start by proving the question for $i = 3$ which is trivial. So, suppose $i = 3$  and we also have $x_1, x_2, x_3 \in C$ and $\theta_1 + \theta_2 + \theta_3 = 1$ with $\theta_1, \theta_2, \theta_3 \geq 0$. 
  	  We will show that $$\theta_1 x_1 + \theta_2 x_2 + \theta_3 x_3 \in C\quad (1)$$ 
  	  There is at least one of the $\theta_i$ that does not equal $1$.
  	  Now, assume, without loss of generality that $\theta_i \neq 1$, where $$\theta_1 + \theta_2 + \theta_3 = 1 \iff $$$$\theta_2 + \theta_3 = 1 - \theta_1 \iff\quad,   (1 - \theta_1 > 0)$$ $$\cfrac{\theta_2 + \theta_3}{1-\theta_1} = 1 \Rightarrow$$ $$1 = m_1 + m_2 = \cfrac{\theta_2 + \theta_3}{1-\theta_1} \Rightarrow$$ $$m_1 = \cfrac{\theta_2}{1-\theta_1}\quad(2)\quad and \quad m_2 = \cfrac{\theta_3}{1-\theta_1}\quad(3)\Rightarrow$$ $$\theta_2 = m_1(1 - \theta_1)\quad and \quad \theta_3 = m_2(1 - \theta_2)$$ 
  	  Thus, from $(1), (2)\quad and\quad (3)$, we have 	$$\theta_1 x_1 + (1 - \theta_1)(m_1 x_1 + m_2 x_3)\quad (4)$$
  	  Since $C$ is convex and $x_2, x_3 \in C$, we conclude that $(m_1 x_1 + m_2 x_3) \in C$ as "extensions". Thus, since both points are in the convex $C$, then the function is in $C$ as well.\par 
  	  \newpage
  	  Now that we have proven that the function belongs to set C for $i = 3$, we can prove with the same steps that it is also true for the general example $i = k$ where $$\theta_1 x_1 + \theta_2 x_2 + \theta_3 x_3 +\quad . . .\quad +\theta_k \in C\quad (5)$$. Following the method above, we find that $$\theta_1 + \theta_2 + ... + \theta_k = 1 \iff$$ $$\theta_2 + \theta_3 +\quad . . .\quad + \theta_k = 1 - \theta_1 \iff \quad,   (1 - \theta_1 > 0)$$ $$\cfrac{\theta_2 + \theta_3 +\quad . . .\quad +\theta_k}{1-\theta_1} = 1 \Rightarrow$$ $$m_1 = \cfrac{\theta_2}{1-\theta_1}\quad (6)\quad . . .\quad m_{k-1} = \cfrac{\theta_k}{1 - \theta_1}\quad (7)$$. As in the proof for $i = 3$ it is easy to see that from the $(5), (6)\quad and\quad (7)$ we have $$\theta_1 x_1 + (1 - \theta_1)(m_1 x_1 + m_2 x_3 +\quad . . .\quad + m_{k-1} x_k)\quad (8)$$. Thus, we show that since $C$ is convex and all points $x_1, x_2, ..., x_k \in C$ then $(m_1 x_1 + m_2 x_3 +\quad . . .\quad + m_{k-1} x_k) \in C$. So the function given is proved to be convex and belong to set $C$.
  	\newpage
  	
%----------------------------Exercise 2 -----------------------
  	
  	\begin{center}
  	\section{}
  	\end{center}\par
  	Show that a set is convex if and only if its intersection with any line is convex.\\
  	\textbf{Solution:}\par
  	We have a convex set $S$. Suppose we have a convex set $A$.\\
  	\textbf{$\Rightarrow$}\\
  	Let's take two points $p_1, p_2 \in S \cup A$, and point $p$ is on the line segment of $p_1p_2$.
  	\begin{center}
  	$\bullet$  $p \in S$ because $S$ is convex\\
  	$\bullet$  $p \in A$ because $A$ is convex\\
  	So $p \in S \cap A$\\
  	\end{center}
  	Since the two points $p_1$ and $p_2$ are arbitrary, we can assume that $S \cap A$ as well. (even in the case where $S \cap A$ is null it is convex) \\
  	\textbf{$\Leftarrow$}\\
  	Let $L \cap C$ which is convex for all lines in $L$. Assume that $C$ is not convex, thus $\exists x,y \in C$, with $\lambda \in [0,1]$, where $$\lambda x + (1 - \lambda)y \notin C$$. But, the line $ax + (1 - a)y, a \in \R$ intersects with $C$, thus, the intersection is convex.\par 
  	The line segment between $x,y$ belongs to the intersection $\lambda x + (1 - \lambda)y$, thus belonging to $C$ as well, which negates our previous assumption, proving that $C$ is indeed convex.
  	\newpage
  	
%----------------------------Exercise 3 ------------------------
  	\begin{center}
  	\section{}
  	\end{center}\par
  	  	Show that a set is affine if and only if its intersection with any line is affine.\\
  	\textbf{Solution:}\par
  	As proved above in question $(3)$, the intersection of convex set $S$ with \underline{any} line is convex.\par
  	Let's take two points $x_1$ and $x_2$, where $x_1, x_2 \in S$. From the proof above, we know that the line between these two points is convex. Thus, the convex combination of $x_1$ and $x_2$ belongs to the intersection and expectedly to $S$. To conclude, since the line belongs to convex set $S$, the set is affine.
  	\newpage

%----------------------------Exercise 4 ------------------------
  	
  	\begin{center}
  	\section{}
  	\end{center}\par
  	  	A set $C$ is midpoint convex, if whenever two points $a, b \in C$, the average or midpoint
$(a + b)/2$ is in $C$. Obviously, a convex set is midpoint convex. Prove that if $C$ is closed and
midpoint convex, then $C$ is convex.\\
  	\textbf{Solution:}\par
  	Let $ x, y \in C$ and assume that z is a point on the line segment between x and y. We construct a sequence {z_i} := $(x_i + y_i)/2$, where x_0 = x, y_0 = y, and \par
  	x_i_+_1 =
 $\begin{cases} 
      z_i & z_i \leq z \\
      x_i &  z_i > z 
   \end{cases}
$
\quad y_i_+_1 = 
\ \begin{cases} 
     z_i & z_i \leq z \\
      y_i &  z_i > z
   \end{cases}
\] \\ 
By this construction, $z$ is always on the line segment between $x_i$ and $y_i$, so we have $||z_i - z|| \leq ||x-y||2^{-i}$, which means $z_i$ converges to $z$. By midpoint convexity of $C$, we have $z_i \in C$ for all i, and because $C$ is closed we have $z \in C$. This is true for all $z$ on the line segment between x and y, so $C$ is convex. 
  	
  	\newpage  	
  	
%----------------------------Exercise 5 ------------------------  	
  	
  	\begin{center}
  	\section{}
  	\end{center}\par
  	  	Show that the convex hull of a set $S$ is the intersection of all convex sets that contain
$S$.(The same method can be used to show that the conic, or affine, or linear hull of a set $S$ is
the intersection of all conic sets, or affine sets, or subspaces that contain $S$.\\
  	\textbf{Solution:}\par
  	Let $A$ be the set of all convex combinations of points in $S$. Since: \\
  	\begin{itemize}
  	\item $S \subseteq Conv(S)$ \\
  	 \item $\forall x \in A$  is a convex combination of points in $Conv(S)$. \\
  	 \end{itemize}
  	 Hence, we have, as proven in question 2, that $A \subseteq Conv(S)$. \\
  	 To prove the reverse inclusion, we show that $A$ is a convex subspace containing $S$. Since $Conv(S)$ is the smallest possible convex subspace that contains $S$, it will show that $A \supseteq Conv(S)$. \\
  	 We have that $S \subseteq A$, and we have to show that $A$ is a convex subspace. \\
  	 Assume two points $x_1, x_2 \in A$. \\
  	 By definition there is: \\
  	 	\begin{itemize}
  	\item $y_1, ..., y_p, y_{p+1}, ..., y_q$\\
  	 \item $a_1, ..., a_p, a_{p+1}, ..., a_q$, such that \\
  	 \item $x_1 = \sum_{i=1}^{p}{a_i}{y_i}$, \quad $x_2 = \sum_{i=p+1}^{q}{a_i}{y_i}$ \quad and \\
  	 \item $\sum_{i=1}^{p}{a_i} = \sum_{i=p+1}^{q}{a_i} =1. $
  	 \end{itemize} 
  	 \\
  	 For any $\beta \in \R$, we have \\
  	 \quad $\overline{x} = \beta x_1 + (1 - \beta) x_2  = \sum_{i=1}^{p}{\beta a_i y_i} + \sum_{i=p+1}^{q}{(1-\beta) a_i y_i},\\ $
  	 with $\sum_{i=1}^{p}{\beta a_i } + \sum_{i=p+1}^{q}{(1-\beta) a_i} = 1 \\ $. 
  	 Thus, we conclude that $\overline{x}$ is a convex combination of points in $S$, and we have $\overline{x} \in A$, showing that A is a convex subset. \\
  	 Now we proved both sides, showing that the convex hull  of a set S is the intersection of all convex sets that contain $S$.
  	\newpage
  	
%----------------------------Exercise 6 ------------------------  	
  	
  	\begin{}
  	\section{}
  	\end{center}\par
  	What is the distance between two parallel hyperplanes $\{x \in \R^n : a^Tx = b_1 \}$ and
$\{x \in \R^n : a^Tx = b_2 \}$?\\
  	\textbf{Solution:}\par
  	Let $x_1$ be a point in the first hyperplane, and consider the line $L$, that goes through $x_1$, in the direction of $a$. The equation of $L$ is $x_1+at, t \in \R$. Now we have to find the intersection of $L$ with the second hyperplane: \\
  	$$a^T(x_1+at) = b2 \quad \Leftrightarrow \quad  t = \frac{b_2 - a^Tx_1}{a^Ta} \quad \Leftrightarrow \quad t = \frac{b_2 - b_1}{a^Ta}$$. \\
  	Therefore, let's call the intersection point $x_2$, which is equal to: \\
  	$$x_2 = x_1 + a\frac{b_2 - b_1}{a^Ta}$$. \\
  	Thus, we only have to find the distance between $x_1$ and $x_2$: \\
  	$$||x_1 - x_2|| = \frac{|b_2-b_1|}{||a||^2}$$.
  	\par
 
  	
  	
  	\newpage
  	
%----------------------------Exercise 7 ------------------------  	
  	
  	\begin{center}
  	\section{}
  	\end{center}\par
  	Let a and b be distinct points in $\R^n$. Show that the set of all points that are closer (in
Euclidean norm) to a than b is a halfspace.\\
  	\textbf{Solution: }\par
  	The euclidean norm is nonnegative, therefore:  \\
  	$$||x-a||_2 \leq ||x-b||_2 \quad \Leftrightarrow \quad ||x-a||_2^2 \leq ||x-b||_2^2 \quad \Leftrightarrow \\$$
  	$$\Leftrightarrow \quad (x-a)^T(x-a) \leq (x-b)^T(x-b) \quad  \Leftrightarrow \\$$
 $$ 	\Leftrightarrow \quad x^Tx - 2a^Tx + a^Ta \leq x^Tx - 2b^Tx + b^Tb \quad \Leftrightarrow \\$$
  $$	\Leftrightarrow \quad 2(b-a)^Tx \leq b^Tb - a^Ta \\ $$
  	Therefore, the set is indeed halfspace, if we substitute $ c = 2(b-a)$ and $ d = b^Tb - a^Ta$ .
  	\newpage
  	
\end{document}