\documentclass[12pt]{article}
\usepackage{amsmath}
\usepackage{amssymb}
\usepackage[utf8]{inputenc}
\usepackage[english]{babel}
\newcommand{\R}{\mathbb{R}}

\title{Algorithmic Operation Research \\ Homework 2}
\date{25-10-2019}
\author{Theodora Panagea - 1115201400135 \\ Anna-Aikaterini Kavvada - 1115201500050}

\begin{document}
	\maketitle{}
  	\pagenumbering{arabic}
  	
%-------------------Exercise 1--------------------------

\subsection*{Exercise 1}
Find a differentiable function $f: \R \rightarrow \R$ such that $f$ does not have an extremum at its critical point. \\
\textbf{Solution:} \par

\newpage

%-------------------Exercise 2--------------------------

\subsection*{Exercise 2}
Given a positive integer $S$, which decompositions  \\
$$a_1+\cdots+a_n = S$$ \\
with the $a_i$ positive integers have the largest product $a_1 \cdots a_n$ ? \\
\textbf{Solution:} \par

\newpage

%-------------------Exercise 3--------------------------

\subsection*{Exercise 3}
Find the optimal solution to the Diet Problem when the cost function is \\
$$Cost(x_1, x_2) = x_1 + x_2.$$
\textbf{Solution:} \par

\newpage

%-------------------Exercise 4--------------------------

\subsection*{Exercise 4}
 Let $A, B \in \R^{nxn}$. Show that the traditional way of computing their product $AB$ requires a total of $(2n-1)n^2$ arithmetic operations.\\
\textbf{Solution:} \par

\newpage

%-------------------Exercise 5--------------------------

\subsection*{Exercise 5}
Consider the problem of solving a system of $n$ linear equations in  $n$ unknowns. Show that the Gaussian elimination method requires $\mathcal{O}(n^3)$ arithmetic operations in order to either compute a solution or to decide that no solution exist.\\
\textbf{Solution:} \par

\newpage

%-------------------Exercise 6--------------------------

\subsection*{Exercise 6}
Suppose that we are given a set of vectors in $\R^n$ that form a basis and let $y$ be an $n$ arbitrary vector in $\R^n$. We wish to express $y$ as a linear combination of the basis vectors. How can this by accomplished?\\
\textbf{Solution:} \par

\newpage

%-------------------Exercise 7--------------------------

\subsection*{Exercise 7}
Study the paper with title: "Do dogs know Calculus?" found in the Readings folder.\\
\textbf{Solution:} \par


\end{document}  	